\documentclass[12pt,]{article}
%\usepackage{nips15submit_e,times}
\usepackage{url}

\providecommand{\tightlist}{%
  \setlength{\itemsep}{0pt}\setlength{\parskip}{0pt}}

\usepackage{lmodern}
\usepackage{amssymb,amsmath}
\usepackage{ifxetex,ifluatex}
\usepackage{fixltx2e} % provides \textsubscript
\ifnum 0\ifxetex 1\fi\ifluatex 1\fi=0 % if pdftex
  \usepackage[T1]{fontenc}
  \usepackage[utf8]{inputenc}
\else % if luatex or xelatex
  \ifxetex
    \usepackage{mathspec}
    \usepackage{xltxtra,xunicode}
  \else
    \usepackage{fontspec}
  \fi
  \defaultfontfeatures{Mapping=tex-text,Scale=MatchLowercase}
  \newcommand{\euro}{€}
\fi
% use upquote if available, for straight quotes in verbatim environments
\IfFileExists{upquote.sty}{\usepackage{upquote}}{}
% use microtype if available
\IfFileExists{microtype.sty}{%
\usepackage{microtype}
\UseMicrotypeSet[protrusion]{basicmath} % disable protrusion for tt fonts
}{}
\usepackage{color}
\usepackage{fancyvrb}
\newcommand{\VerbBar}{|}
\newcommand{\VERB}{\Verb[commandchars=\\\{\}]}
\DefineVerbatimEnvironment{Highlighting}{Verbatim}{commandchars=\\\{\}}
% Add ',fontsize=\small' for more characters per line
\usepackage{framed}
\definecolor{shadecolor}{RGB}{248,248,248}
\newenvironment{Shaded}{\begin{snugshade}}{\end{snugshade}}
\newcommand{\KeywordTok}[1]{\textcolor[rgb]{0.13,0.29,0.53}{\textbf{{#1}}}}
\newcommand{\DataTypeTok}[1]{\textcolor[rgb]{0.13,0.29,0.53}{{#1}}}
\newcommand{\DecValTok}[1]{\textcolor[rgb]{0.00,0.00,0.81}{{#1}}}
\newcommand{\BaseNTok}[1]{\textcolor[rgb]{0.00,0.00,0.81}{{#1}}}
\newcommand{\FloatTok}[1]{\textcolor[rgb]{0.00,0.00,0.81}{{#1}}}
\newcommand{\ConstantTok}[1]{\textcolor[rgb]{0.00,0.00,0.00}{{#1}}}
\newcommand{\CharTok}[1]{\textcolor[rgb]{0.31,0.60,0.02}{{#1}}}
\newcommand{\SpecialCharTok}[1]{\textcolor[rgb]{0.00,0.00,0.00}{{#1}}}
\newcommand{\StringTok}[1]{\textcolor[rgb]{0.31,0.60,0.02}{{#1}}}
\newcommand{\VerbatimStringTok}[1]{\textcolor[rgb]{0.31,0.60,0.02}{{#1}}}
\newcommand{\SpecialStringTok}[1]{\textcolor[rgb]{0.31,0.60,0.02}{{#1}}}
\newcommand{\ImportTok}[1]{{#1}}
\newcommand{\CommentTok}[1]{\textcolor[rgb]{0.56,0.35,0.01}{\textit{{#1}}}}
\newcommand{\DocumentationTok}[1]{\textcolor[rgb]{0.56,0.35,0.01}{\textbf{\textit{{#1}}}}}
\newcommand{\AnnotationTok}[1]{\textcolor[rgb]{0.56,0.35,0.01}{\textbf{\textit{{#1}}}}}
\newcommand{\CommentVarTok}[1]{\textcolor[rgb]{0.56,0.35,0.01}{\textbf{\textit{{#1}}}}}
\newcommand{\OtherTok}[1]{\textcolor[rgb]{0.56,0.35,0.01}{{#1}}}
\newcommand{\FunctionTok}[1]{\textcolor[rgb]{0.00,0.00,0.00}{{#1}}}
\newcommand{\VariableTok}[1]{\textcolor[rgb]{0.00,0.00,0.00}{{#1}}}
\newcommand{\ControlFlowTok}[1]{\textcolor[rgb]{0.13,0.29,0.53}{\textbf{{#1}}}}
\newcommand{\OperatorTok}[1]{\textcolor[rgb]{0.81,0.36,0.00}{\textbf{{#1}}}}
\newcommand{\BuiltInTok}[1]{{#1}}
\newcommand{\ExtensionTok}[1]{{#1}}
\newcommand{\PreprocessorTok}[1]{\textcolor[rgb]{0.56,0.35,0.01}{\textit{{#1}}}}
\newcommand{\AttributeTok}[1]{\textcolor[rgb]{0.77,0.63,0.00}{{#1}}}
\newcommand{\RegionMarkerTok}[1]{{#1}}
\newcommand{\InformationTok}[1]{\textcolor[rgb]{0.56,0.35,0.01}{\textbf{\textit{{#1}}}}}
\newcommand{\WarningTok}[1]{\textcolor[rgb]{0.56,0.35,0.01}{\textbf{\textit{{#1}}}}}
\newcommand{\AlertTok}[1]{\textcolor[rgb]{0.94,0.16,0.16}{{#1}}}
\newcommand{\ErrorTok}[1]{\textcolor[rgb]{0.64,0.00,0.00}{\textbf{{#1}}}}
\newcommand{\NormalTok}[1]{{#1}}
\ifxetex
  \usepackage[setpagesize=false, % page size defined by xetex
              unicode=false, % unicode breaks when used with xetex
              xetex]{hyperref}
\else
  \usepackage[unicode=true]{hyperref}
\fi
\hypersetup{breaklinks=true,
            bookmarks=true,
            pdfauthor={},
            pdftitle={Introductory statistics with intRo},
            colorlinks=true,
            citecolor=blue,
            urlcolor=blue,
            linkcolor=magenta,
            pdfborder={0 0 0}}
\urlstyle{same}  % don't use monospace font for urls
\setlength{\parindent}{0pt}
\setlength{\parskip}{6pt plus 2pt minus 1pt}
\setlength{\emergencystretch}{3em}  % prevent overfull lines
\setcounter{secnumdepth}{5}

\newcommand{\fix}{\marginpar{FIX}}
\newcommand{\new}{\marginpar{NEW}}
%\nipsfinalcopy

%\pdfminorversion=4
% NOTE: To produce blinded version, replace "0" with "1" below.

\title{\bf Introductory statistics with \texttt{intRo}}
\author{Eric Hare \\ Iowa State University \\ \texttt{}  and \\ Andee Kaplan \\ Iowa State University \\ \texttt{} }
\date{}

\usepackage{graphicx}
\usepackage{multirow}
\usepackage{color}
\usepackage{float}

\pdfminorversion=4

\newcommand{\ak}[1]{{\color{magenta} #1}}
\bigskip


% DON'T change margins - should be 1 inch all around.
\addtolength{\oddsidemargin}{-.5in}%
\addtolength{\evensidemargin}{-.5in}%
\addtolength{\textwidth}{1in}%
\addtolength{\textheight}{1.3in}%
\addtolength{\topmargin}{-.8in}%

\begin{document}

\def\spacingset#1{\renewcommand{\baselinestretch}%
{#1}\small\normalsize} \spacingset{1}


\maketitle


\begin{abstract}
\texttt{intRo} is a web-based application for performing basic data
analysis and statistical routines. Leveraging the power of R and Shiny,
\texttt{intRo} implements common statistical functions in an extensible
modular structure, while including a point-and-click interface for the
novice statistician. This simplicity lends itself to a natural
presentation in an introductory statistics course as a substitute for
other commonly used statistical software packages. \texttt{intRo} is
currently deployed at the URL \url{http://www.intro-stats.com}. Within
this paper, we introduce the application and explore the design
decisions underlying \texttt{intRo}, as well as highlight some
challenges and advantages of reactive programming.
\end{abstract}

\noindent%
{\it Keywords:}  Interactivity, Programming Paradigms, Reactive Programming,
Reproducibility, Statistical Software
\vfill

\newpage
\spacingset{1.45} % DON'T change the spacing!


\section{Introduction}\label{introduction}

The widespread adoption of R (R Core Team 2014) as a tool for
statistical analysis has undoubtedly been an important development for
the scientific community. However, using R in most cases still requires
a basic knowledge of programming concepts which may pose a steep
learning curve for the introductory statistics student (Tan, Ting, and
Ling 2009). This additional time commitment may explain why introductory
courses often utilize point-and-click applications, even if the
instructor himself/herself uses R in their own work. Still, some
compromises must be made when using many graphical applications,
including dealing with software licenses and unsupported desktop
platforms.

In teaching Introduction to Business Statistics at Iowa State
University, we witnessed profound struggles by students attempting to
practice introductory concepts discussed in class using current
software. Scrimshaw (2001) notes in his manuscript that ``open-ended
packages, like any others, may create obstacles to learning simply
through their lack of user-friendliness in the sheer mechanics of
operating them, rather than any intrinsic difficulty in the
content\ldots{}.'' In our own experience teaching, students' struggles
were often directly related to the use of the software and not any sort
of fundamental misunderstanding of the material, in agreement with
Scrimshaw's finding.

Multiple software packages have recently been written in an attempt to
spur interest in R programming and statistics. DataCamp's (DataCamp
2014) courses are user-friendly ways to learning basic R programming and
data analysis techniques. Swirl (Carchedi et al. 2014) is a similar
interactive tool to make learning R more fun by learning it within R
itself. Project MOSAIC (Pruim, Kaplan, and Horton 2014) has created a
suite of tools to simplify the teaching of statistics in the form of an
R package. The primary goal of DataCamp and Swirl is to teach R
programming, rather than facilitate the learning of introductory
statistics. Project MOSAIC's goal is to faciliate this learning, but
using the package requires a knowledge of R programming that the
introductory student may not have. R Commander (Fox 2005) and Deducer
(Fellows 2012) provide a graphical front-end to many statistical
functions in R, but are not web-based and thus require local
installation and configuration. iNZight Lite (Wild 2015) also attempts
to expose students to data analysis without requiring programming
knowledge, but does not include reproducible R code, and therefore has
less of a focus on spurring interest in coding for students.

Upon the release of RStudio's Shiny (RStudio and Inc. 2014) it became
easier for an R-based analysis to be converted to an interactive web
application. This in turn led us to create an introductory statistics
application which we call \texttt{intRo}, available at
\url{http://www.intro-stats.com}. \texttt{intRo} offers a number of key
advantages over traditional statistics software, including ease of
access and an aim to foster student interest in coding. Attempting to
entirely hide the programming aspect from students, even in introductory
classes, is a lost opportunity to get students interested in statistical
computing. It is also a lost opportunity reaching students who learn
differently or have a computational background. Another advantage is its
modular structure, which allows course instructors to tailor the
application towards the needs of a particular class, rather than accept
a piece of software as is. Additionally, \texttt{intRo} stands apart
from new tools in that it is a supplement to an existing class, fully
usable by a beginning statistics student. These advantages will be
discussed at length in this text. The paper is structured as follows:
section \ref{what-is-intro} introduces the application, its features and
its usability, and provides motivations for the design decisions
underlying it. Section \ref{understanding-intro} provides technical
details on the design decisions of \texttt{intRo}, in particular, its
modular and extensible structure. Section
\ref{deploying-intro-instances} describes the process of deploying a
classroom instance of \texttt{intRo} tailored to the individual
instructor. Finally, section \ref{conclusions-and-future-work} discusses
some future possibilities and limitations of the software.

\section{\texorpdfstring{What is
\texttt{intRo}?}{What is intRo?}}\label{what-is-intro}

\texttt{intRo} is a web-based tool to accompany an introductory
statistics class. It is meant to assist in the learning of statistics
rather than as a stand-alone delivery of statistics education, with the
intention of being used in conjunction with a guided class. An
accompanying R package, titled \texttt{intRo} and available on GitHub,
assists in the downloading, running, and deploying of \texttt{intRo}
instances.

Three fundamental philosophies that guided the creation of
\texttt{intRo}. In particular, \texttt{intRo} is \emph{easy} to use and
can be an \emph{exciting} part of learning statistics. Additionally,
\texttt{intRo} is an \emph{extensible} tool, allowing for a course
instructor using \texttt{intRo} to tailor the tool for his or her own
classroom needs.

In the development of \texttt{intRo}, we focused on aspects of the user
interface (UI) and output that make it easy to pick up without extensive
training. We used large, easy to click icons in the page header to help
students find what they need more easily. We also made the functionality
available the minimal necessary for an introductory statistics course.
Figure \ref{fig:user_experience} illustrates the simple steps a student
takes to generate a result in \texttt{intRo}. In this instance, a
student clicks on the Graphical tab to create a mosaic plot. The student
sees the plot, and elects to click the save button to store the plot and
its corresponding code to the final compendium.

\begin{figure}[H]
\centering
\includegraphics[width=\linewidth]{user_experience.pdf}
\caption{A typical student experience of generating a result in \texttt{intRo}. In this instance, a student clicks on the Graphical tab to create a mosaic plot. The student sees the plot, and elects to click the save button to store the plot (and its corresponding code) to the final compendium.}
\label{fig:user_experience}
\end{figure}

Beyond being simple, \texttt{intRo} is also consistent. The tool is
organized around specific tasks a student may perform in the process of
a data analysis, called modules. To the student, a module is simply a
page of statistics functionality that maintains a consistent layout,
helping the student to become familiar with the location of the options,
the results, and the code. Figure \ref{fig:ui} highlights the five
elements that comprise the \texttt{intRo} interface.

\begin{figure*}[ht!]
\centering
\includegraphics[width=\linewidth]{ui_annotate.pdf}
\caption{The five elements that comprise the \texttt{intRo} application: 1) top navigation, 2) side navigation, 3) options panel, 4) results panel, and 5) code panel.}
\label{fig:ui}
\end{figure*}

\begin{enumerate}
\def\labelenumi{\arabic{enumi}.}
\tightlist
\item
  \textbf{Top Navigation} - The top navigation bar includes two sets of
  clickable icons. The left-aligned buttons are informational buttons.
  The first is a link to \texttt{intRo}. The second is a link to the
  documentation page. The third is a link to the GitHub repository where
  the code for \texttt{intRo} is housed. The final button is a link to
  our websites, which contain contact information if there are any
  questions or comments. The right-aligned buttons are \texttt{intRo}
  utilities. The first is a link to toggle the visibility of the code
  panel (5). The middle icon downloads an rmarkdown (Allaire et al.
  2014) document of the analysis performed. The last is a link to print
  the stored module results, and the associated code (if visible).
\item
  \textbf{Side Navigation} - The side navigation panel includes a list
  of data analysis tasks.
\item
  \textbf{Options Panel} - The options panel includes task-specific
  options which the student can use to customize their results.
\item
  \textbf{Results Panel} - The results pane displays the result of the
  selected module and options.
\item
  \textbf{Code Panel} - The code panel displays the R code used to
  generate the results from the student's \texttt{intRo} session. The
  code panel is shown by default to facilitate a transition to coding,
  but can be hidden by clicking the code toggle button in the Top
  Navigation bar.
\end{enumerate}

The modules included in \texttt{intRo} are split into three higher level
categories - data, summaries, and inference. Under each of these
categories, there are seven default modules, which perform specific data
analysis tasks that employ an easy to use point-and-click interface.
More modules can easily be added by an instructor, as detailed in
section \ref{modules}. The default modules support uploading and
downloading a dataset, transforming variables, graphical and numerical
summaries, simple linear regression, contingency tables, and T-tests.

We've also created a documentation website that is consistent with the
interface of the application. The documentation is available at
\url{http://gammarama.github.io/intRo} and hyperlinked within the
application itself. The documentation describes the specific
functionality provided by the default \texttt{intRo} modules.

Statistics courses with a lab component can most benefit from the use of
\texttt{intRo}. As students learn the aforementioned concepts, they can
practice using the tools to perform the statistical tests and summaries.
They simply upload a dataset and click the module appropriate to the
section of the class. Once they choose the options desired by the
instructor (variables, confidence levels, etc.), they can view the
results and ultimately print the output for turning in. Early labs can
walk the student through the limited number of button presses and clicks
necessary, while later labs can more vaguely describe the intended
output in hopes that the student can easily navigate the \texttt{intRo}
interface to obtain the results.

\texttt{intRo} has an ulterior motive as well: to get students excited
about programming. By navigating about the user interface of
\texttt{intRo}, students are actually creating a fully-executable R
script that they can download and run locally as well as viewing the
script change real-time within the application. This code creation
element of \texttt{intRo} is meant to generate excitement about
programming in R and empower students to feel that they can generate
code as well. \texttt{intRo} uses rmarkdown's render function in order
to print the results, by dymanically executing the student's R script.
By default, the output will include the R code, but if the student
elects to hide the source code by clicking the code toggle button at the
top, the code will not appear in the printed results.

This reproducibility framework has three key advantages. First, this
eases a student who may be intimidated by programming into the idea that
interacting with a user interface is really just a frontend for code.
Seeing the correspondence between graphical clicks and printed code will
hopefully also lessen the fear of coding that many students may have.
Second, an analysis created by \texttt{intRo} can be reproduced in an R
session to easily assess and extend the results. Finally, ``printing''
the results of an \texttt{intRo} analysis amounts to nothing more than
executing the R code on the server, adding another layer of
reproducibility. These concepts are important because they encourage
best practices with regards to disclosure of analysis methods in
research (Baggerly and Berry 2011; Xie 2015).

On the front end, user interaction with \texttt{intRo} is split into
bitesize chunks that we call modules. Modules are self-contained pieces
of functionality which implement common statistical procedures that are
often used in introductory statistics classes. These modules form the
core of \texttt{intRo} and are discussed at length in the next section.

\section{\texorpdfstring{Understanding
\texttt{intRo}}{Understanding intRo}}\label{understanding-intro}

To fully understand \texttt{intRo}, it is important to expand upon two
underlying concepts. The first is the idea of modularity in the context
of a Shiny application. The second is the idea of reactive programming
in the context of a web application.

\subsection{Modules}\label{modules}

An \texttt{intRo} module is a set of self-contained executible R scripts
that together produce a set of introductory statistics functionality.
\texttt{intRo} modules were designed in this way to allow for simple
dynamic creation of the user interface at run-time, as well as ease the
process of converting existing analysis code to the \texttt{intRo}
framework. A high-level diagram of this process is given in figure
\ref{fig:app_creation_modules}. \texttt{intRo} modules are split up into
multiple R scripts which are included either in Shiny's user interface
or server definitions. At runtime, the \texttt{intRo} sources in the
specified modules (contained in the modules folder) to dynamically
generate the functionality available in the application. This allows for
the specific functionality needed to be determined and adjusted by the
individual course instructor. In this example, the instructor is
electing to include a nonparametric module, which is not enabled by
default, to allow the students to perform a wilcoxon rank sum test.

\begin{figure*}[ht!]
\centering
\includegraphics[width=\linewidth]{app_creation_modules.pdf}
\caption{This figure depicts how the Shiny \texttt{server.R} and \texttt{ui.R} files are populated using the modular structure within \texttt{intRo}. \texttt{intRo} modules are split up into multiple R scripts which are included either in Shiny's user interface or server definitions. At runtime, the \texttt{intRo} sources in the specified modules (contained in the modules folder) to dynamically generate the functionality available in the application. This allows for the specific functionality needed to be determined and adjusted by the individual course instructor. In this example, the instructor is electing to include a nonparametric module, which is not enabled by default, to allow the students to perform a wilcoxon rank sum test.}
\label{fig:app_creation_modules}
\end{figure*}

Section \ref{dynamic-ui-generation} provides some technical details on
how we implemented this. For the rest of this section, we focus on the
structure and development of the modules themselves, to aid in the
process of creating and deploying new modules.

Modularity was a design decision we focused on from the start of
\texttt{intRo\textquotesingle{}s} development. There are some practical
benefits to thinking of related statistics and data science
functionality in terms of modules. Course instructors can easily enable
or disable functionality depending on the exact needs of the particular
class being taught. Because this is done at run-time, including new
modules is as simple as downloading and placing it within
\texttt{intRo\textquotesingle{}s} modules folder, or removing existing
modules from that folder. Furthermore, errors can be more easily
isolated to specific components. For instance, if an error is
encountered, simply disabling the module can provide a temporary
workaround while the issue is identified. Finally, modularity helps to
organize the different pieces of code into functionality chunks that
make it easier for developers to maintain.

\texttt{intRo} modules are not to be confused with Shiny modules (Cheng
2015). Shiny modules are a recent feature added to Shiny which allows
the bundling of inputs and outputs into a single set of functionality.
They are more general and suitable for any application. \texttt{intRo}
modules are specifically for statistics functionality and work within
the \texttt{intRo} application only.

An \texttt{intRo} module consists of the following scripts:

\begin{itemize}
\tightlist
\item
  \emph{helper.R} - R code that performs some statistical analysis or
  transformation. This would typically be in the form of a function, and
  similar to any standard R script.
\item
  \emph{libraries.R} - Code to load any libraries which are not part of
  core R.
\item
  \emph{observe.R} - Shiny observer code typically used to update
  choices of an input box.
\item
  \emph{output.R} - Shiny output code defining the results of the
  analysis that should be displayed to the student.
\item
  \emph{reactive.R} - Shiny reactives, typically containing data that
  depend on inputs.
\item
  \emph{ui.R} - Shiny user interface definition, including the placement
  of the inputs and outputs.
\end{itemize}

The modules provided with \texttt{intRo} are contained in the modules
folder. The top level directory in the modules folder defines the
category of the module (currently \texttt{data}, \texttt{summaries}, or
\texttt{inference}). Within each of these categories is a folder named
according to the name of the module. This folder houses the previously
defined scripts. As an example, we will walk through the contents of the
\texttt{nonparametric} module as previously mentioned in this section.

Since the \texttt{nonparametric} module performs a statistical test, it
is part of the inference category, and hence can be found in the
\texttt{intRo} repository at \textbf{modules/inference/nonparametric}.
Let's first examine \emph{helper.R}:

\begin{Shaded}
\begin{Highlighting}[]
\NormalTok{nonparametrictest <-}\StringTok{ }\NormalTok{function(intro.data, x, y, }
                                \NormalTok{conflevel, althyp, hypval) \{}
    \KeywordTok{interpolate}\NormalTok{(~(}\KeywordTok{wilcox.test}\NormalTok{(}\DataTypeTok{x =} \NormalTok{df$x, }\DataTypeTok{y =} \NormalTok{df$y, }
                              \DataTypeTok{conf.level =} \NormalTok{conf, }
                              \DataTypeTok{alternative =} \NormalTok{althyp, }
                              \DataTypeTok{mu =} \NormalTok{hypval)),}
                  \DataTypeTok{df =} \KeywordTok{quote}\NormalTok{(intro.data),}
                  \DataTypeTok{x =} \NormalTok{x,}
                  \DataTypeTok{y =} \NormalTok{y,}
                  \DataTypeTok{conf =} \NormalTok{conflevel,}
                  \DataTypeTok{althyp =} \NormalTok{althyp,}
                  \DataTypeTok{hypval =} \NormalTok{hypval,}
                  \DataTypeTok{mydir =} \NormalTok{userdir, }
                  \StringTok{`}\DataTypeTok{_env}\StringTok{`} \NormalTok{=}\StringTok{ }\KeywordTok{environment}\NormalTok{(), }
                  \DataTypeTok{file =} \StringTok{"code_nonparametric.R"}\NormalTok{)}
\NormalTok{\}}
\end{Highlighting}
\end{Shaded}

This script is most immediately similar to standard R code. In this
case, a function \texttt{nonparametrictest} is created which, depending
on the values of the parameters, ultimately returns the result of a
wilcoxon rank sum test. One important difference from a typical R script
is that each call in the script is wrapped in a function called
\texttt{interpolate} (Wickham 2015). \texttt{interpolate} both executes
the given R code on the server, and also writes the code executed to the
script window at the bottom of \texttt{intRo}.

Because all the code needed to implement a wilcoxon rank sum test is
found in the \texttt{base} and \texttt{stats} package, the
\texttt{libraries.R} file is empty for the \texttt{nonparametric}
module. Additionally, there are no reactive objects defined, so
\texttt{reactive.R} is also empty. \texttt{observe.R}, which defines the
Shiny observers needed, is reproduced below:

\begin{Shaded}
\begin{Highlighting}[]
\KeywordTok{observe}\NormalTok{(\{}
    \KeywordTok{updateSelectizeInput}\NormalTok{(session, }\StringTok{"group1_non"}\NormalTok{, }
                         \DataTypeTok{choices =} \KeywordTok{intro.numericnames}\NormalTok{(), }
                         \DataTypeTok{selected =} \KeywordTok{ifelse}\NormalTok{(}\KeywordTok{checkVariable}\NormalTok{(}
                             \KeywordTok{intro.data}\NormalTok{(), input$group1_non), }
                             \NormalTok{input$group1_non, }
                             \KeywordTok{intro.numericnames}\NormalTok{()[}\DecValTok{1}\NormalTok{]))}
    \KeywordTok{updateSelectizeInput}\NormalTok{(session, }\StringTok{"group2_non"}\NormalTok{, }
                         \DataTypeTok{choices =} \KeywordTok{intro.numericnames}\NormalTok{(), }
                         \DataTypeTok{selected =} \KeywordTok{ifelse}\NormalTok{(}\KeywordTok{checkVariable}\NormalTok{(}
                             \KeywordTok{intro.data}\NormalTok{(), input$group2_non), }
                             \NormalTok{input$group2_non, }
                             \KeywordTok{intro.numericnames}\NormalTok{()[}\DecValTok{2}\NormalTok{]))}
\NormalTok{\})}

\KeywordTok{observeEvent}\NormalTok{(input$store_nonparametric, \{}
    \KeywordTok{cat}\NormalTok{(}\KeywordTok{paste0}\NormalTok{(}\StringTok{"}\CharTok{\textbackslash{}n\textbackslash{}n}\StringTok{"}\NormalTok{, }\KeywordTok{paste}\NormalTok{(}\KeywordTok{readLines}\NormalTok{(}
        \KeywordTok{file.path}\NormalTok{(userdir, }\StringTok{"code_nonparametric.R"}\NormalTok{)), }
        \DataTypeTok{collapse =} \StringTok{"}\CharTok{\textbackslash{}n}\StringTok{"}\NormalTok{)), }
        \DataTypeTok{file =} \KeywordTok{file.path}\NormalTok{(userdir, }\StringTok{"code_All.R"}\NormalTok{), }
        \DataTypeTok{append =} \OtherTok{TRUE}\NormalTok{)}
\NormalTok{\})}
\end{Highlighting}
\end{Shaded}

Shiny observers are a class of reactive objects within the Shiny
paradigm which do not return a value (RStudio and Inc. 2014). For
further discussion of reactivity, see section
\ref{reactive-design-choices}. In this example, observers are created to
ensure that the choices of variable for the \texttt{nonparametric}
module are only numeric variables. This is accomplished by utilizing the
global reactive \texttt{intro.numericnames()}, which returns a character
vector containing the variables in the current dataset that are numeric.
Finally, there is an event observer to store code generated from the
module into the overall code script upon clicking the store button. The
presence of this observer code and the definition of the button in the
user interface are enforced, and must be present in any \texttt{intRo}
module.

The \texttt{output.R} code is very simple:

\begin{Shaded}
\begin{Highlighting}[]
\NormalTok{output$nonparametrictest <-}\StringTok{ }\KeywordTok{renderPrint}\NormalTok{(\{}
    \KeywordTok{return}\NormalTok{(}\KeywordTok{nonparametrictable}\NormalTok{(}\KeywordTok{intro.data}\NormalTok{(), input$group1_non, }
                              \NormalTok{input$group2_non, input$conflevel_non, }
                              \NormalTok{input$althyp_non, input$hypval_non))}
\NormalTok{\})}
\end{Highlighting}
\end{Shaded}

The \texttt{output.R} script then simply uses Shiny's
\texttt{renderPrint} function to display the resulting table.

Finally, the \texttt{ui.R} code is shown below:

\begin{Shaded}
\begin{Highlighting}[]
\NormalTok{nonparametric_ui <-}\StringTok{ }\KeywordTok{tabPanel}\NormalTok{(}\StringTok{"Nonparametric"}\NormalTok{,}
    \KeywordTok{column}\NormalTok{(}\DecValTok{4}\NormalTok{,}
         \KeywordTok{wellPanel}\NormalTok{(}
             \KeywordTok{selectizeInput}\NormalTok{(}\StringTok{"group1_non"}\NormalTok{, }\DataTypeTok{label =} \StringTok{"Group 1 (x)"}\NormalTok{, }
                            \DataTypeTok{choices =} \KeywordTok{numericNames}\NormalTok{(mpg), }
                            \DataTypeTok{selected =} \KeywordTok{numericNames}\NormalTok{(mpg)[}\DecValTok{1}\NormalTok{]),}
             \KeywordTok{selectizeInput}\NormalTok{(}\StringTok{"group2_non"}\NormalTok{, }\StringTok{"Group 2 (y)"}\NormalTok{, }
                            \DataTypeTok{choices =} \KeywordTok{numericNames}\NormalTok{(mpg), }
                            \DataTypeTok{selected =} \KeywordTok{numericNames}\NormalTok{(mpg)[}\DecValTok{2}\NormalTok{]),}
             
             \KeywordTok{hr}\NormalTok{(),}
             
             \KeywordTok{selectizeInput}\NormalTok{(}\StringTok{"althyp_non"}\NormalTok{, }\StringTok{"Alternative Hypothesis"}\NormalTok{, }
                            \KeywordTok{c}\NormalTok{(}\StringTok{"Two-Sided"} \NormalTok{=}\StringTok{ "two.sided"}\NormalTok{, }
                              \StringTok{"Greater"} \NormalTok{=}\StringTok{ "greater"}\NormalTok{, }\StringTok{"Less"} \NormalTok{=}\StringTok{ "less"}\NormalTok{)),}
             \KeywordTok{numericInput}\NormalTok{(}\StringTok{"hypval_non"}\NormalTok{, }\StringTok{"Hypothesized Value"}\NormalTok{,}
                          \DataTypeTok{value =} \DecValTok{0}\NormalTok{),}
             \KeywordTok{sliderInput}\NormalTok{(}\StringTok{"conflevel_non"}\NormalTok{, }\StringTok{"Confidence Level"}\NormalTok{,}
                         \DataTypeTok{min=}\FloatTok{0.01}\NormalTok{, }\DataTypeTok{max=}\FloatTok{0.99}\NormalTok{, }\DataTypeTok{step=}\FloatTok{0.01}\NormalTok{, }\DataTypeTok{value=}\FloatTok{0.95}\NormalTok{),}
             
             \KeywordTok{hr}\NormalTok{(),}
             
             \NormalTok{tags$}\KeywordTok{button}\NormalTok{(}\StringTok{""}\NormalTok{, }\DataTypeTok{id =} \StringTok{"store_nonparametric"}\NormalTok{, }\DataTypeTok{type =} \StringTok{"button"}\NormalTok{, }
                         \DataTypeTok{class =} \StringTok{"btn action-button"}\NormalTok{, }\KeywordTok{list}\NormalTok{(}\KeywordTok{icon}\NormalTok{(}\StringTok{"save"}\NormalTok{), }
                         \StringTok{"Store Nonparametric Result"}\NormalTok{), }
                         \DataTypeTok{onclick =} \StringTok{"$('#top-nav a:has(> .fa-print, }
\StringTok{                         .fa-code, .fa-download)').highlight();"}\NormalTok{)}
         \NormalTok{)}
    \NormalTok{),}
    
    \KeywordTok{column}\NormalTok{(}\DecValTok{8}\NormalTok{,}
         \NormalTok{tags$}\KeywordTok{b}\NormalTok{(}\StringTok{"Nonparametric Results"}\NormalTok{),}
         \KeywordTok{verbatimTextOutput}\NormalTok{(}\StringTok{"nonparametrictest"}\NormalTok{)}
    \NormalTok{)}
\NormalTok{)}
\end{Highlighting}
\end{Shaded}

This script defines all the inputs and outputs that the student will
see. The only requirements from \texttt{intRo\textquotesingle{}s}
perspective are (1) that there exist a store button at the bottom of the
middle panel for storing the results of the analysis in the code script,
and (2) that configuration options appear in the width 4 column in the
middle, and output appears in the width 8 column on the right. The
remaining input and output definitions depend on the statistical
analysis or transformation being performed.

Although the structure of an \texttt{intRo} module is relatively
straightforward, producing the code needed in a more seamless fashion
would certainly help open up the creation of such modules to a wider
audience. As we discuss in the conclusions and future work section,
providing an \texttt{intRo} module creation tool to abstract away some
of the less common coding paradigms, like the use of
\texttt{interpolate}, is an important effort that will continue to be
pursued.

\subsection{Reactive design choices}\label{reactive-design-choices}

Reactive programming is a programming paradigm that ``tackles issues
posed by event-driven applications by providing abstractions to express
programs as reactions to external events and having the language
automatically manage the flow of time (by conceptually supporting
simultaneity), and data and computation dependencies'' (Bainomugisha et
al. 2012). As implemented by Shiny, results automatically update when
users interact with the interface.

intRo leverages the reactive programming nature of Shiny, and as such is
designed around the idea of user input cascading through the entire
application. In a typical Shiny application, users interact with inputs
that act as parameters to function, which in turn yield different
results. Within \texttt{intRo}, the students are able to interact with
and manipulate the data underlying the entire application. This posed
many challenges in the creation of \texttt{intRo} and drove design
decisions, namely timely save points according to the student's
workflow, and reactive updating of variable lists tied to inputs across
the entire application. Because the student may experiment with
different configurations or select different variables, we did not want
to store all actions taken in the intRo session. Rather, each module
includes a button allowing the student to explicitly store the output
visible in the results panel into the R script. This way, output is only
stored when the student is satisfied, and the resulting output is not
cluttered with unnecessary information.

In the creation of \texttt{intRo} we walked a fine line between giving
the student flexibility and having realistic usability. At the same
time, \texttt{intRo} was created as a consumer of another package,
Shiny, in which we as developers were the beneficiaries of another team
of developers' decision to balance flexibility and usability. For a
tangible example, consider the graphical summaries module. We only allow
variables of a type consistent with the selected plot to be displayed.
This is a conscious decision that limits an \texttt{intRo} user's
flexibility, while maximizing the usability (by minimizing crashes) of
the application. On the flip side of this, Shiny allows much higher
flexibility. For instance, the entire application (including user
interface) is created dynamically upon load, based on the modules
currently housed within \texttt{intRo}. However, Shiny does have limits
on its flexibility based on the designers decisions for usability. One
current example is the slider element. This element allows for fixed
width steps from its minimum to its maximum. The JavaScript library
being utilized in Shiny allows for arbitrary function calls to to
generate these steps, however they must be written in plain JavaScript.
This is an example of a decision made by the developers of Shiny to
limit functionality in favor of usability of their package.

\section{\texorpdfstring{Deploying \texttt{intRo}
Instances}{Deploying intRo Instances}}\label{deploying-intro-instances}

Although students can access intRo from
\url{http://www.intro-stats.com}, course instructors may wish to
download, customize, and deploy their own instance, perhaps with new
modules or modified theming or functionality. \texttt{intRo} can be
downloaded, ran, and deployed on ShinyApps.io through the use of the R
package \texttt{intRo}. Currently, the package is only available on
GitHub, and can be installed using the devtools package as follows:

\begin{Shaded}
\begin{Highlighting}[]
\NormalTok{devtools::}\KeywordTok{install_github}\NormalTok{(}\StringTok{"gammarama/intRo"}\NormalTok{)}
\end{Highlighting}
\end{Shaded}

After installing the \texttt{intRo} package, the first function one
should call is \texttt{download\_intRo}. \texttt{download\_intRo} takes
as an argument a directory in which to store the application. By
default, it selects the working directory of the R session. This
function clones the application branch of the \texttt{intRo} repository
on GitHub, and hence will pull the latest version of the code whenever
it is ran.

Running \texttt{download\_intRo} will produce an \texttt{intRo} folder
in the specified folder. It can then be ran as any Shiny application,
using Shiny's \texttt{runApp} command. However, we have provided a
wrapper function \texttt{run\_intRo} which adds some additional
customization options to the execution process. \texttt{run\_intRo}
takes as argument the path to the folder containing the \texttt{intRo}
application. It also takes several more optional arguments:

\begin{itemize}
\tightlist
\item
  \texttt{enabled\_modules}: A character vector containing the modules
  to enable
\item
  \texttt{theme}: A string representing a shinythemes theme to use
\item
  \texttt{...}: Additional arguments passed to Shiny's \texttt{runApp}
  function
\end{itemize}

The package provides help documentation which explains in further detail
the format that these arguments would take, but as an example, suppose I
wanted to download \texttt{intRo} to my working directory, execute an
\texttt{intRo} session with only the data sources, data transform, and
numerical summaries modules enabled, and apply the cerulean theme. The
series of calls to do so would be as follows:

\begin{Shaded}
\begin{Highlighting}[]
\KeywordTok{download_intRo}\NormalTok{()}
\KeywordTok{run_intRo}\NormalTok{(}\DataTypeTok{enabled_modules =} \KeywordTok{c}\NormalTok{(}\StringTok{"data/transform"}\NormalTok{, }\StringTok{"summaries/numerical"}\NormalTok{), }
          \DataTypeTok{theme =} \StringTok{"cerulean"}\NormalTok{)}
\end{Highlighting}
\end{Shaded}

Note that the data sources module is required, and hence must be
included in all intRo sessions and need not be specified in the
enabled\_modules argument.

If the intent is to use a specific instance of \texttt{intRo} where many
students will access it at the same time, such as in an introductory
statistics class, it may be preferable to deploy a custom instance of
\texttt{intRo} to a publicly accessible URL. The package provides a
function \texttt{deploy\_intRo} which is a wrapper for the
\texttt{deployApp} function contained in the shinyapps package. Once the
shinyapps package is installed and configured, \texttt{deploy\_intRo}
will upload \texttt{intRo} as an application on the instructor's
ShinyApps.io account. The function takes the same arguments as
\texttt{run\_intRo}, so it can be deployed with a custom selection of
modules, and a customized theme. It also takes an additional argument
\texttt{google\_analytics}, which allows the specification of a Google
Analytics tracking ID. It also takes \texttt{...} as additional
arguments to be passed into the \texttt{deployApp} routine. For example,
if we wished to deploy the instance of \texttt{intRo} we ran previously,
we would call it like so:

\begin{Shaded}
\begin{Highlighting}[]
\KeywordTok{deploy_intRo}\NormalTok{(}\DataTypeTok{enabled_modules =} \KeywordTok{c}\NormalTok{(}\StringTok{"data/transform"}\NormalTok{, }\StringTok{"summaries/numerical"}\NormalTok{), }
             \DataTypeTok{theme =} \StringTok{"cerulean"}\NormalTok{)}
\end{Highlighting}
\end{Shaded}

Once the process finished, the app will become available at
\url{http://<user>.shinyapps.io/intRo}, where
\texttt{\textless{}user\textgreater{}} is the username of the
ShinyApps.io account configured.

\section{Conclusions and future work}\label{conclusions-and-future-work}

\texttt{intRo} can be a powerful and effective tool for introductory
statistics education. Its modular structure allows it to be flexible
enough for many different applications and curriculums. Its ease-of-use
allows the student to focus her attention on the statistics task at
hand, rather than struggling with software licenses and confusing
interface navigation. Reproducible code generated from each analysis can
be used to spark an interest in R programming in those who might
otherwise not be exposed to it.

In addition to the current functionality, there are some practical
improvements in the works that will make \texttt{intRo} more useful to
both students and instructors. In particular, we have begun development
on an R package which will allow \texttt{intRo} modules to be created
automatically from user written R code. This package will generate the
necessary file structure to allow the module's incorporation into
\texttt{intRo} as well as translate user code to \texttt{intRo}
compatible code and populate the necessary files. This will vastly
improve \texttt{intRo\textquotesingle{}s} flexibility and allow it to be
used in a wider range of curricula, including more advanced statistics
courses. Additionally, we would like to utilize the expanded interactive
capabilities of ggvis in order to make \texttt{intRo's} plots more
engaging to students. One way to do this would be implementing linked
plots, in which interactions with one plot are reflected in other plots
that illustrate the same data. This would be particularly useful in the
regression module so that students could explore observations with high
influence and leverage.

We hope to use \texttt{intRo} in courses to collect feedback regarding
the ease of use and functionality. This will allow us to assess its
usefulness relative to software used in the past, as well as gauge areas
for improvement. Furthermore, we can determine the effectiveness of code
printing on generating excitement from the students about programming in
R.

Challenges do exist with regards to the wider adoption of
\texttt{intRo}. For instance, we will need to monitor how well the
server hosting \texttt{intRo} handles the load of dozens of students
performing data analyses at once. If performance issues are encountered,
the infrastructure used may need to be expanded to handle current and
future load. An unknown quantity will be how feasible it is to increase
adoption of \texttt{intRo} across Iowa State, as well as to other
universities. One limitation of \texttt{intRo} is that uploading a
dataset beyond about 30,000 rows tends to be slow. Even once the data is
successfully uploaded, the default modules produce results more slowly
than with smaller datasets. This is a limitation that should be further
investigated if and when \texttt{intRo} sees wider adoption.

Regardless, tools that focus on usability and extensibility in
statistics education, such as \texttt{intRo}, are sure to encourage the
next round of innovators to be interested and excited about statistical
computing.

\section{Supplementary material}\label{supplementary-material}

All code and documents related to this manuscript are available at
\url{https://github.com/gammarama/intRo}.

\section{Appendix}\label{appendix}

\subsection{Dynamic UI Generation}\label{dynamic-ui-generation}

\texttt{intRo\textquotesingle{}s} user interface and functionality is
dynamically generated depending on the set of modules enabled. The key
driver to populating \texttt{server.R} and \texttt{ui.R} is the modules
folder, the directory structure of which defines the placement of each
module. The interface is then created with the following statement.

\footnotesize

\begin{Shaded}
\begin{Highlighting}[]
\NormalTok{## Source ui}
\NormalTok{mylist <-}\StringTok{ }\KeywordTok{list}\NormalTok{()}
\NormalTok{old_heading <-}\StringTok{ ""}
\NormalTok{for (i in }\KeywordTok{seq_along}\NormalTok{(modules)) \{}
    \NormalTok{my.module <-}\StringTok{ }\KeywordTok{strsplit}\NormalTok{(modules[i], }\StringTok{"/"}\NormalTok{)[[}\DecValTok{1}\NormalTok{]]}
    \NormalTok{if (my.module[}\DecValTok{1}\NormalTok{] !=}\StringTok{ }\NormalTok{old_heading) \{}
        \NormalTok{mylist[[}\KeywordTok{length}\NormalTok{(mylist) +}\StringTok{ }\DecValTok{1}\NormalTok{]] <-}\StringTok{ }\NormalTok{Hmisc::}\KeywordTok{capitalize}\NormalTok{(my.module[}\DecValTok{1}\NormalTok{])}
        \NormalTok{old_heading <-}\StringTok{ }\NormalTok{my.module[}\DecValTok{1}\NormalTok{]}
    \NormalTok{\}}
    \NormalTok{mylist[[}\KeywordTok{length}\NormalTok{(mylist) +}\StringTok{ }\DecValTok{1}\NormalTok{]] <-}\StringTok{ }\KeywordTok{get}\NormalTok{(}\KeywordTok{paste}\NormalTok{(my.module[}\DecValTok{2}\NormalTok{], }
        \StringTok{"ui"}\NormalTok{, }\DataTypeTok{sep =} \StringTok{"_"}\NormalTok{))}
\NormalTok{\}}

\NormalTok{## mylist is a list containing the different ui}
\NormalTok{## module code Create the UI}
\KeywordTok{shinyUI}\NormalTok{(}\KeywordTok{navbarPage}\NormalTok{(}\StringTok{"intRo"}\NormalTok{, }\DataTypeTok{id =} \StringTok{"top-nav"}\NormalTok{, }\DataTypeTok{theme =} \StringTok{"bootstrap.min.css"}\NormalTok{, }
    \KeywordTok{tabPanel}\NormalTok{(}\DataTypeTok{title =} \StringTok{""}\NormalTok{, }\DataTypeTok{icon =} \KeywordTok{icon}\NormalTok{(}\StringTok{"home"}\NormalTok{), }\KeywordTok{fluidRow}\NormalTok{(}\KeywordTok{do.call}\NormalTok{(navlistPanel, }
        \KeywordTok{c}\NormalTok{(}\KeywordTok{list}\NormalTok{(}\DataTypeTok{id =} \StringTok{"side-nav"}\NormalTok{, }\DataTypeTok{widths =} \KeywordTok{c}\NormalTok{(}\DecValTok{2}\NormalTok{, }\DecValTok{10}\NormalTok{)), }
            \NormalTok{mylist)))), ...))}
\end{Highlighting}
\end{Shaded}

\normalsize

The key piece of code being the \texttt{do.call} statement loading the
list of ui elements from the module's \texttt{ui.R} file. The server
functions are then dynamically generated using a similar method.

\footnotesize

\begin{Shaded}
\begin{Highlighting}[]
\KeywordTok{shinyServer}\NormalTok{(function(input, output, session) \{}
    \NormalTok{types <-}\StringTok{ }\KeywordTok{c}\NormalTok{(}\StringTok{"helper.R"}\NormalTok{, }\StringTok{"observe.R"}\NormalTok{, }\StringTok{"reactive.R"}\NormalTok{, }
        \StringTok{"output.R"}\NormalTok{)}
    
    \NormalTok{modules_tosource <-}\StringTok{ }\KeywordTok{file.path}\NormalTok{(}\StringTok{"modules"}\NormalTok{, }\KeywordTok{apply}\NormalTok{(}\KeywordTok{expand.grid}\NormalTok{(modules, }
        \NormalTok{types), }\DecValTok{1}\NormalTok{, paste, }\DataTypeTok{collapse =} \StringTok{"/"}\NormalTok{))}
    
    \NormalTok{for (mod in modules_tosource) \{}
        \KeywordTok{source}\NormalTok{(mod, }\DataTypeTok{local =} \OtherTok{TRUE}\NormalTok{)}
    \NormalTok{\}}
\NormalTok{\})}
\end{Highlighting}
\end{Shaded}

\normalsize
In this way, we were able to have \texttt{intRo} be fully extensible,
its structure and functionality dependent entirely on the modules
present within the application.

\section*{References}\label{references}
\addcontentsline{toc}{section}{References}

\hypertarget{refs}{}
\hypertarget{ref-rmarkdown}{}
Allaire, JJ, Jonathan McPherson, Yihui Xie, Hadley Wickham, Joe Cheng,
and Jeff Allen. 2014. \emph{Rmarkdown: Dynamic Documents for R}.
\url{http://CRAN.R-project.org/package=rmarkdown}.

\hypertarget{ref-baggerly2011reproducible}{}
Baggerly, Keith A, and Donald A Berry. 2011. ``Reproducible Research.''
\emph{AMSTAT News: The Membership Magazine of the American Statistical
Association}, no. 403. American Statistical Association: 16--17.

\hypertarget{ref-bainomugisha2012survey}{}
Bainomugisha, Engineer, Andoni Lombide Carreton, Tom Van Cutsem, Stijn
Mostinckx, and Wolfgang De Meuter. 2012. ``A Survey on Reactive
Programming.'' In \emph{ACM Computing Surveys}. Citeseer.

\hypertarget{ref-swirl}{}
Carchedi, Nick, Bill Bauer, Gina Grdina, and Sean Kross. 2014.
\emph{Swirl: Learn R, in R.}
\url{http://CRAN.R-project.org/package=swirl}.

\hypertarget{ref-shinymodules}{}
Cheng, Joe. 2015. ``Shiny - Modularizing Shiny App Code.''
\url{http://shiny.rstudio.com/articles/modules.html}.

\hypertarget{ref-datacamp}{}
DataCamp. 2014. ``Online R Tutorials and Data Science Courses -
Datacamp.'' \url{https://www.datacamp.com/}.

\hypertarget{ref-fellows2012}{}
Fellows, Ian. 2012. ``Deducer: A Data Analysis Gui for R.''
\emph{Journal of Statistical Software} 49 (8).

\hypertarget{ref-fox2005}{}
Fox, John. 2005. ``The R Commander: A Basic-Statistics Graphical User
Interface to R.'' \emph{Journal of Statistical Software} 14 (9).

\hypertarget{ref-mosaic}{}
Pruim, Randall, Daniel Kaplan, and Nicholas Horton. 2014. \emph{Mosaic:
Project Mosaic (Mosaic-Web.org) Statistics and Mathematics Teaching
Utilities}. \url{http://CRAN.R-project.org/package=mosaic}.

\hypertarget{ref-r-stat}{}
R Core Team. 2014. \emph{R: A Language and Environment for Statistical
Computing}. Vienna, Austria: R Foundation for Statistical Computing.
\url{http://www.R-project.org/}.

\hypertarget{ref-shiny}{}
RStudio, and Inc. 2014. \emph{Shiny: Web Application Framework for R}.
\url{http://CRAN.R-project.org/package=shiny}.

\hypertarget{ref-scrimshaw2001computers}{}
Scrimshaw, Peter. 2001. ``Computers and the Teacher's Role.''
\emph{Knowledge, Power and Learning}. London, Paul Chapman Publishing
Ltd.

\hypertarget{ref-5359977}{}
Tan, P. H., C. Y. Ting, and S. W. Ling. 2009. ``Learning Difficulties in
Programming Courses: Undergraduates' Perspective and Perception.'' In
\emph{Computer Technology and Development, 2009. Icctd '09.
International Conference on}, 1:42--46.
doi:\href{https://doi.org/10.1109/ICCTD.2009.188}{10.1109/ICCTD.2009.188}.

\hypertarget{ref-interpolate}{}
Wickham, Hadley. 2015. ``Graphics \& Computing Student Paper Winners @
Jsm 2015.'' \url{https://github.com/hadley/15-student-papers}.

\hypertarget{ref-inzight}{}
Wild, Chris. 2015. ``INZight Lite.''
\url{http://lite.docker.stat.auckland.ac.nz}.

\hypertarget{ref-xie2015}{}
Xie, Yihui. 2015. \emph{Dynamic Documents with R and Knitr}. Vol. 29.
CRC Press.

\end{document}
